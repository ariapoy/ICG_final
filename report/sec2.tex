\section{Problem definition}
Our goal is to transform the photos/images into \textbf{pencil sketch} style.
In this project, we focus on three scenarios:
\begin{tasks}
    \item Beginner sketching: flower, avatar, and life photo of team members.
    \item Scene sketching: day and night of buildings, such as 101, NTU.
    \item Video sketching: NTU College of Social Sciences building, mother and daughter, couples in the sunshine.
\end{tasks}
The first task is also the guide of the beginner in the art course. And we know \textbf{human hand sketching} has the following features:
\begin{enumerate}
    \item Shape of lines is shaken.
    \item Long lines are drawn divided into short strokes.
    \item Variation in line brightness.
    \item Main contours are drawn with multiple lines/several times.
\end{enumerate}
An example of human-like sketch is shown in \figref{fig4} \cite{2017_Okawa_canny}.
\begin{figure}
    \centering
    \includegraphics[width=0.3\textwidth]{image/fig4.png}
    \caption{An example of hand-written sketch}
    \label{fig4}
\end{figure}

The second task is advanced skills in sketching. Urban sketching \figref{fig4-1} is about observing the world, witnessing, and recording. Here are some tips, pay attention to the silhouette and ensure the angle and sight with proportionally accurate. Then emphasize the light source on different objects, such as buildings, clouds, background, and things that appeal to you.
\begin{figure}
    \centering
    \includegraphics[width=\textwidth]{image/fig4-1.png}
    \caption{An example of urban sketching}
    \label{fig4-1}
\end{figure}

The last task is a special case. It is only the thing that computers can do. We could try the pencil-style effects of continuous photos/images than merge them into a video \figref{fig4-2}.
\begin{figure}
    \centering
    \includegraphics[width=\textwidth]{image/fig4-2.png}
    \caption{An example of video sketching}
    \label{fig4-2}
\end{figure}
